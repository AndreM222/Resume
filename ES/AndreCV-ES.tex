
\documentclass[10pt, letterpaper]{article}
\usepackage[
    pdftitle={Andre Mossi CV (ES)},
    pdfauthor={Andre Mossi},
    pdfcreator={LaTeX and NeoVim}
]{hyperref}

\usepackage{geometry}
\geometry{
    a4paper,
    left=12.7mm,
    right=12.7mm,
    top=12.7mm,
    bottom=12.7mm
}

\usepackage{titlesec}
\titlespacing{\section} {0pt}{-0.1cm}{0cm}
\titlespacing{\subsection} {0pt}{-0.3cm}{0cm}

\pagestyle{empty}

\begin{document}

\begin{center}
    \textbf{\huge{Roberto (Andre) Mossi Milla}}\\

    \vspace{0.3cm}

    \small{
        Erie, Pennsylvania, 16506\\
        \href{tel:+18147901591}{814-790-1591}\\
        \href{mailto:mossiroberto0392@gmail.com}{mossiroberto0392@gmail.com}\\
        \href{andremossi-linktree.vercel.app}{andremossi-linktree.vercel.app}
    }
    \vspace{0.5cm}
\end{center}

\begin{Form}
    \textbf{\section*{SUMMARY}}
    \normalsize{
        Desarrollador Full-Stack cursando una Licenciatura en Ciencias de la Computación. He desarrollado proyectos
        paralelos, tomado clases en línea y estudiado en la Universidad de Gannon con el objetivo de aumentar mis
        conocimientos.
    }

    \textbf{\section*{EXPERIENCIA LABORAL}}

    \textbf{\subsection*{Dinant, Tegucigalpa, Honduras}}

    \subsubsection*{Desarrollador Full-Stack, Junio-Julio 2023}
    \begin{itemize}
        \item
            Gestioné una base de datos de miles de clientes utilizando SQL, y se trabaje en una aplicación para
            visualizar e interactuar con los datos. Entre las muchas bases de datos, una de ellas se diseñó para
            llevar un seguimiento de todos los detalles de la maquinaria, como el estado actual del equipo de varias
            empresas con las que trabajamos, y para encontrar asistencia fácilmente si era necesario. Otra fue una base
            de datos utilizada para mantener información sobre compras de productos, estado y tipo de plan, para tener
            una interacción más rápida y efectiva con los clientes.

        \item
            Created web apps utilizing Apex Oracle, JavaScript, CSS, and some HTML for business clients that are
            prominent latino sellers. The purpose of one of them was to generate digital contracts with easiness
            by updating in real time possible answers in the required filling fields as the user makes choices and
            based on users’ permissions. Another one was an assistance app meant for factory workers we were
            affiliated with to ask for help. Based on the users profile and filled up fields, the program will
            determine the best possible helpers they may need for their necessities.

    \end{itemize}

    \textbf{\subsection*{Gannon University, Pennsylvania, Estados Unidos}}

    \subsubsection*{Asistente de Investigación, Presente}
    \begin{itemize}
        \item
            Trabaje con un profesor desarrollando una investigación sobre una simulación de organismos destinada
            a enfatizar la selección natural bajo restricciones ambientales.

        \item
            Utilizando tecnologías como Python, Cuda y Pytorch, trabajamos para dar vida a estos organismos con
            inteligencia artificial, para estudiar su comportamiento bajo una asignación de supervivencia,
            en este caso un juego de las traes.
    \end{itemize}

    \subsubsection*{Técnico Estudiantil, Agosto - Diciembre 2022}
    \begin{itemize}
        \item
            Trabajé en el departamento de TI de la universidad. Fui responsable de brindar asistencia en problemas
            relacionados con la tecnología a estudiantes o profesores 5 días a la semana.
    \end{itemize}

    \textbf{\subsection*{Yakiniku Byakutan, Tokyo, Japon}}

    \subsubsection*{Personal, Diciembre 2022}
    \begin{itemize}
        \item
            Asistente de cocina. Me encargaba de asegurar que la limpieza del restaurante estuviera conforme
            con los estándares del registro de salud.
    \end{itemize}

    \textbf{\section*{EDUCACIÓN}}

    \textbf{\subsection*{Gannon University, Pennsylvania, Estados Unidos}}
    Bachelor of Science in Computer Systems Engineering, Presente

    \textbf{\subsection*{Harvard University, En Linea}}
    CS50’s Introduction to Computer Science, Diciembre 2022

    \textbf{\subsection*{Extreme Networks, En Linea}}
    Introduction to Future Networks, Mayo 2022

    \textbf{\section*{Additional Skills}}
    \begin{itemize}
        \item Competente en PowerShell, Bash, and NVIM.

        \item
            Aprendí React, Next.js, Vite y Vercel para el despliegue y desarrollo de mi portafolio y sitio web
            tipo link tree.

        \item Nivel intermedio en C++.

        \item
            Aprendí Linux e hice una configuración personalizada ("rice") con el propósito de aumentar mi
            conocimiento sobre sistemas operativos, entender cómo funciona una máquina y ser capaz de leer
            el código de otras personas.

        \item Fluidez escrita y oral en español.

        \item Fluidez escrita y oral en ingles.

    \end{itemize}

    \textbf{\section*{RESUMEN DE CUALIFICACIONES}}

    \begin{itemize}
        \item 3 años de experiencia en Windows PowerShell.

        \item 3 años de experiencia creando juegos en Unreal Engine.

        \item 1 año de experiencia depurando software utilizando Visual Studio, IntelliJ, NVIM y VSCode.

        \item 1 año de experiencia desarrollando software en C++.

        \item 5 meses de experiencia desarrollando aplicaciones web en JavaScript, HTML y CSS.

        \item 3 meses de experiencia desarrollando software en Java.

        \item 3 meses de experiencia desarrollando software en C\# y C.

        \item 2 meses de experiencia desarrollando servicios en la nube a gran escala.

        \item 2 meses de experiencia trabajando con Oracle y Oracle Apex.

        \item 2 meses de experiencia desarrollando sitios web en NodeJS, React y Next.js.

        \item 3 meses de experiencia desarrollando software en Python.
    \end{itemize}

    \textbf{\section*{PREMIOS Y HONORES}}

    \begin{itemize}
        \item "Premio al Desarrollo Escolar" obtenido por el destacado desarrollo de un videojuego usando Blueprints en Unreal Engine 4, 2020.

        \item "Premio al Desarrollo Escolar" obtenido por el destacado desarrollo de un videojuego utilizando C++ y Unreal Engine 4, 2016.
    \end{itemize}


\end{Form}
\end{document}
